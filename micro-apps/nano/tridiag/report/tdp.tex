\documentclass[10pt,twocolumn]{article}
\usepackage[top=0.5in,bottom=0.7in,left=0.5in,right=0.5in]{geometry}
\setlength{\parskip}{0in}
\setlength{\parskip}{1ex plus 0.5ex minus 0.2ex}
\usepackage{graphicx}
\usepackage{subfigure}
\begin{document}
\sloppy

\begin{center}
  {\bf Solving diagonally dominant tridiagonal systems in batch for SHOC and HOMMEXX} \\
  E3SM/SCREAM (\today)
\end{center}

\noindent {\bf Motivation.}
At least two components of E3SM solve a linear system in each vertical column of the model.
SHOC and the theta-l nonhydrostatic dycore of HOMME,
for which a performance portable implementation in HOMMEXX is in progress,
use an implicit time step in the vertical direction.
In each case, the matrix is tridiagonal and strictly diagonally dominant.
Strict diagonal dominance implies pivoting is not needed for numerical stability.
SHOC has multiple right-hand sides (RHS); HOMME has just one.
HOMME computes at the level of elements.
With $n_p = 4$, there are 16 systems to solve per Newton step.
(One could filter out converged columns to get $\le 16$.)

Vendors provide solvers for this problem.
MKL extends LAPACK with {\tt ?dttrfb} and {\tt ?dttrsb}, the factorization and solver routine for exactly this problem;
LAPACK provides only {\tt ?gttrf} and {\tt ?gttrs}, which pivot.
The solver routines support multiple RHS,
but the RHS must be column-major, so efficient vectorization along the RHS is not supported.
cuSPARSE provides {\tt cusparse?gtsv2StridedBatch} and {\tt cusparse?gtsvInterleavedBatch}.
(The second is not considered because its interface is too far away from our problem formats.)
These are kernels; they cannot be called from a kernel.
They support only the case of one RHS.
They require extra memory to be allocated proportional to the batch and system sizes.

Among these restrictions, the most important is that cu\-SPARSE's routines are kernels.
This means to call them, one must collect all the data in an application kernel,
end the kernel, call the cuSPARSE kernel, then start a new application kernel.
Some GPU developers advocate for many small kernels,
but we find large kernels generally are better;
on non-GPU computers, large kernels essentially always are better, if written correctly.
The other restrictions present performance opportunities.

This report evaluates SCREAM-written solvers.
They remove all the restrictions.
Thus, we consider three solver groups: cuSPARSE, MKL, and SCREAM.
First, we discuss algorithms.
Second, we demonstrate that SCREAM solvers reach roughly performance parity with,
and in some cases are robustly better than,
the vendor solvers on the problem format that all three solver groups support optimally.
Third, we show results for the SHOC problem.
Fourth, we show results for the HOMME problem.

\noindent {\bf Algorithms.}
The Thomas algorithm is Gaussian elimination without pivoting,
with arithmetic arranged to optimize the special case of a tridiagonal matrix.
On GPU, as is standard practice, some form of cyclic reduction (CR) must be used.
We implemented cyclic reduction with a special case for the final one or two levels of reduction.
GPU hardware threads are dispatched across the L/RHS before the rows.
CR is work inefficient---it uses more operations than a serial solver---but
it permits solving a set of rows independently in $2 \log_2 \mathrm{nextpow2}(n) - 1$ sweeps.
Its work inefficiency means we cannot expect to get speedup over optimally implemented CPU solvers as large as
in the case of work efficient computations, such as simple evaluation of level-independent physics parameterization formulas.
So-called parallel CR (PCR) halves the number of sweeps at the cost of substantially more work inefficiency.
We evaluated PCR, and in special cases it is faster than CR.
But in practice CR will always be faster, so we do not consider PCR further.

Algorithm names in the legends have the following scheme.
The main solvers are {\tt cusparse}, for {\tt cusparse?gtsv2StridedBatch};
the MKL solvers as already named;
{\tt thomas}, for the Thomas-type solvers;
and {\tt cr}, for the cyclic reduction solvers.
The suffix {\tt pack} is sometimes specified when a scalar version is being distinguished from the pack ({\tt scream::pack}) version.
The suffix {\tt a$\{$1,m$\}$x$\{$1,m$\}$} is sometimes specified when an implementation specific to one ({\tt 1}) or many ({\tt m}) LHS ({\tt a}) or RHS ({\tt x}) is being distinguished.
These suffixes appear mostly in the appendix.
The implementations overload on Kokkos array rank, so the suffixes do not appear at the interface level.

\noindent {\bf Methods.}
We use the following procedure to collect data.
For a given problem and solver configuration,
the solver is run on the batch once for burn-in and then once for timing.
The problem is run at least 25 times, and the median of these instances is reported.
For V100 SCREAM solvers,
the number of warps is chosen for each problem size to get the best result;
cuSPARSE internally selects this parameter.
SKX is run with all 96 hardware threads.
KNL is run with 136 and 272 threads,
and the best result from these two settings is selected for each point.
Generally, this selection procedure favors MKL solvers over SCREAM ones:
SCREAM solvers rely mostly on the vector processing units (VPU), and there are only two per core;
for this problem type, MKL solvers rely mostly on threading, and there are four threads per core.
MKL's serial libraries are linked to avoid inner OpenMP dispatch,
which would hurt MKL performance on batch solves.

All figures are formatted as follows.
The $x$ axis is number of linear systems to solve in batch.
The $y$ axis is number of systems solved per second.
Red curves are for Nvidia Volta (V100);
green, Intel 48-core Skylake (SKX);
blue, Intel 68-core Knights Landing (KNL).
Solid lines are for SCREAM solvers; dashed, vendor solvers.

\noindent {\bf Performance parity.}
All three solver groups optimally solve a monolithic batch of systems, each system having one RHS.
Figure \ref{fig:one}(a) shows results for this problem format.
(The appendix shows many more variations;
figures \ref{fig:one}, \ref{fig:shoc} are intended to show the most important results with the least amount of data.)

On V100, the SCREAM CR solver is uniformly faster than the cuSPARSE one.
On SKX and KNL, the SCREAM Thomas solver is at parity with (sometimes a little faster, sometimes a little slower) the MKL solver specialized for diagonally dominant systems.
(The appendix shows that both solvers specialized for diagonal dominance are substantially faster than the general tridiagonal solver.)
On SKX, both MKL and SCREAM get a performance boost at very large column counts.
We speculate, but with little confidence, that this is due to Intel Turbo Boost.
It is definitely not a cache effect.
In any case, the problem count is too high to be useful.
In the appendix, some SKX curves show what is likely the L3 cache effect at small problem counts,
which is useful.

On V100, comparing {\tt cr-a1x1p} with {\tt thomas} shows the speedup cyclic reduction provides over the Thomas algorithm,
about a factor of 2--4.
For more RHS than one, the Thomas implementation uses one hardware thread per RHS;
thus, when a problem format has many RHS, Thomas can be more competitive.

A detail concerns Kokkos.
Performance analysis on V100 showed that the Kokkos Cuda backend was choosing
to prefer shared over L1 use of the cache,
even when the user side of the kernel does not use shared memory,
which is a performance bug.
We opened Kokkos issue 2066 with a suggested patch to fix this problem.
This patch improves V100 results by a fair bit (as a second test, the patch also improves HOMMEXX performance) and was used in producing the data for this report.
However, the raw Cuda implementation (solver {\tt cr-a1x1p}, with {\tt p} suffix) is still a bit faster than the Kokkos implementations;
this slight gap is generally expected and is a small cost of performance portability.

\noindent {\bf HOMME problem.}
Figure \ref{fig:one}(b) shows results for the HOMME problem format,
in which there are 16 L/RHS sides per problem.
The vendor solvers are set up to run on the full collection of equations (number of problems times number of L/RHS),
which means they can perform as fast as possible.
In contrast, the SCREAM solvers are dispatched as they will be used in practice:
one 16-L/RHS problem per Kokkos team.
On SKX and KNL, explicit vectorization via packing in the L/RHS direction enables
substantial speedup over MKL.
On V100, SCREAM is robustly faster than cuSPARSE.
On SKX and KNL, spikes in performance occur at element counts such that the number of cores divides the element count, i.e., when the work per core is exactly balanced.

\noindent {\bf SHOC problem.}
SHOC solves systems having multiple RHS, either 2 or 3 plus the number of tracers.
SCREAM will initially have 10 tracers and later approximately 40.
Figure \ref{fig:shoc} shows the three cases of 2, 13, and 43 RHS.
As in the HOMME problem, the vendor solvers are used in the configuration that is best for them,
while the SCREAM solvers are used as they will be in practice.
On V100, SCREAM solvers are again uniformly faster than cuSPARSE.
On V100 and at 43 RHS, the Thomas solver is competitive with CR when there are enough columns.
On SKX, SCREAM is a little slower than MKL at 2 RHS but a little faster at more except at very large column count;
we can summarize this as that SCREAM and MKL are roughly at parity on SKX.
On KNL, SCREAM is uniformly substantially faster than MKL.

\noindent {\bf Summary.} SCREAM solvers solve all problem formats optimally,
exploiting the speedup certain formats enable;
they perform at parity or better relative to vendor solvers;
they use no extra memory, unlike vendor solvers;
and they can be called from kernels, unlike cuSPARSE, with no restrictions on Kokkos team and thread configurations.

\onecolumn

\begin{figure}[hbt]
  \centering
  \subfigure[]{\includegraphics[width=0.48\linewidth]{baseline-prec2-nlev128-simple.pdf}}
  \subfigure[]{\includegraphics[width=0.48\linewidth]{hommexx-prec2-nlev128-nrhs16-simple.pdf}}
  \caption{(a) Baseline comparison against vendor library routines.
  (b) HOMME problem.} \label{fig:one}
\end{figure}
\begin{figure}[hbt]
  \centering
  \subfigure[]{\includegraphics[width=0.48\linewidth]{shoc-prec2-nlev128-nrhs2-simple.pdf}}
  \subfigure[]{\includegraphics[width=0.48\linewidth]{shoc-prec2-nlev128-nrhs13-simple.pdf}} \\
  \subfigure[]{\includegraphics[width=0.48\linewidth]{shoc-prec2-nlev128-nrhs43-simple.pdf}}
  \caption{SHOC problem with three different numbers of RHS.} \label{fig:shoc}
\end{figure}
\clearpage

\noindent {\bf Appendix}

This appendix shows results for many solvers and various problem size parameters.

\begin{figure}[hbt]
  \centering
  \includegraphics[width=0.48\linewidth]{baseline-prec2-nlev72-all.pdf}
  \includegraphics[width=0.48\linewidth]{baseline-prec2-nlev128-all.pdf} \\
  \includegraphics[width=0.48\linewidth]{baseline-prec2-nlev256-all.pdf}
\end{figure}

\begin{figure}[hbt]
  \centering
  \includegraphics[width=0.48\linewidth]{hommexx-prec2-nlev72-nrhs16.pdf}
  \includegraphics[width=0.48\linewidth]{hommexx-prec2-nlev128-nrhs16.pdf} \\
  \includegraphics[width=0.48\linewidth]{hommexx-prec2-nlev256-nrhs16.pdf}
\end{figure}

\begin{figure}[hbt]
  \centering
  \includegraphics[width=0.48\linewidth]{hommexx-prec2-nlev72-nrhs10.pdf}
  \includegraphics[width=0.48\linewidth]{hommexx-prec2-nlev128-nrhs10.pdf} \\
  \includegraphics[width=0.48\linewidth]{hommexx-prec2-nlev256-nrhs10.pdf}
\end{figure}

\begin{figure}[hbt]
  \centering
  \includegraphics[width=0.48\linewidth]{shoc-prec2-nlev72-nrhs2.pdf}
  \includegraphics[width=0.48\linewidth]{shoc-prec2-nlev128-nrhs2.pdf} \\
  \includegraphics[width=0.48\linewidth]{shoc-prec2-nlev256-nrhs2.pdf}
\end{figure}

\begin{figure}[hbt]
  \centering
  \includegraphics[width=0.48\linewidth]{shoc-prec2-nlev72-nrhs13.pdf}
  \includegraphics[width=0.48\linewidth]{shoc-prec2-nlev128-nrhs13.pdf} \\
  \includegraphics[width=0.48\linewidth]{shoc-prec2-nlev256-nrhs13.pdf}
\end{figure}

\begin{figure}[hbt]
  \centering
  \includegraphics[width=0.48\linewidth]{shoc-prec2-nlev72-nrhs43.pdf}
  \includegraphics[width=0.48\linewidth]{shoc-prec2-nlev128-nrhs43.pdf} \\
  \includegraphics[width=0.48\linewidth]{shoc-prec2-nlev256-nrhs43.pdf}
\end{figure}

\end{document}
